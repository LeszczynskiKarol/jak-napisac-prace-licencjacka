% ============================================
% EBOOK: JAK NAPISAĆ PRACĘ LICENCJACKĄ
% Styl akademicki, paleta uniwersytecka
% ============================================

\usepackage{etoolbox}
\usepackage{tikz}
\usetikzlibrary{calc,positioning,shapes.geometric}
\usepackage{xcolor}
\usepackage{fontawesome5}
\usepackage{fancyhdr}
\usepackage{float}
\usepackage{tcolorbox}
\usepackage{tabularx}
\usepackage{colortbl}
\usepackage{array}
\usepackage{enumitem}
\usepackage{graphicx}
\usepackage{hyperref}
\usepackage{tikzsymbols}
\tcbuselibrary{skins,breakable,hooks}

% ============================================
% PALETA KOLOROW - AKADEMICKA
% ============================================

% Kolory główne
\definecolor{acadnavy}{RGB}{24,54,87}
\definecolor{acadgold}{RGB}{184,144,48}
\definecolor{acadburgundy}{RGB}{120,47,64}
\definecolor{acadforest}{RGB}{45,87,64}
\definecolor{acadslate}{RGB}{68,78,88}
\definecolor{acadcopper}{RGB}{156,96,56}
\definecolor{acadplum}{RGB}{88,56,96}
\definecolor{acadteal}{RGB}{40,88,96}

% Kolory jasne (tła)
\definecolor{acadlightnavy}{RGB}{235,240,248}
\definecolor{acadlightgold}{RGB}{255,250,235}
\definecolor{acadlightburgundy}{RGB}{252,240,242}
\definecolor{acadlightforest}{RGB}{238,248,242}
\definecolor{acadlightslate}{RGB}{244,246,248}
\definecolor{acadlightcopper}{RGB}{255,248,240}
\definecolor{acadlightplum}{RGB}{248,242,250}
\definecolor{acadlightteal}{RGB}{238,248,250}

% Aliasy główne
\colorlet{maincolor}{acadnavy}
\colorlet{accentcolor}{acadgold}
\colorlet{warncolor}{acadburgundy}
\colorlet{successcolor}{acadforest}
\colorlet{neutralcolor}{acadslate}

% ============================================
% USUNIECIE STRONY TYTULOWEJ
% ============================================
\AtBeginDocument{\let\maketitle\relax}

% ============================================
% SPIS TRESCI
% ============================================
\fancypagestyle{plain}{
  \fancyhf{}
  \fancyhead[C]{
    \begin{tikzpicture}[remember picture, overlay]
      \fill[acadslate] ([yshift=-12mm]current page.north west) rectangle ([yshift=0mm]current page.north east);
      \node[text=white, anchor=east, font=\bfseries\small] at ([xshift=-15mm, yshift=-6mm]current page.north east) {Spis tre\'sci};
    \end{tikzpicture}
  }
  \fancyfoot[C]{
    \begin{tikzpicture}[remember picture, overlay]
      \fill[acadslate] ([yshift=12mm]current page.south west) rectangle ([yshift=0mm]current page.south east);
    \end{tikzpicture}
  }
  \renewcommand{\headrulewidth}{0pt}
  \renewcommand{\footrulewidth}{0pt}
}

\pretocmd{\tableofcontents}{\pagestyle{plain}\pagenumbering{gobble}}{}{}
\apptocmd{\tableofcontents}{\clearpage\pagenumbering{arabic}\setcounter{page}{1}\pagestyle{main}}{}{}

\renewcommand{\cleardoublepage}{\clearpage}
\KOMAoptions{open=any}

\RedeclareSectionCommand[
  beforeskip=0pt,
  afterskip=1em
]{chapter}

% ============================================
% MARGINESY
% ============================================
\setlength{\headheight}{14mm}
\setlength{\headsep}{4mm}
\setlength{\footskip}{16mm}

% ============================================
% STYLE NAGLOWKOW I STOPEK
% ============================================

% Styl główny
\fancypagestyle{main}{
  \fancyhf{}
  \fancyhead[C]{
    \begin{tikzpicture}[remember picture, overlay]
      \fill[acadnavy] ([yshift=-14mm]current page.north west) rectangle ([yshift=0mm]current page.north east);
      \node[text=acadgold, anchor=west, font=\small] at ([xshift=15mm, yshift=-7mm]current page.north west) {\faIcon{graduation-cap}};
      \node[text=white, anchor=east, font=\bfseries\small] at ([xshift=-15mm, yshift=-7mm]current page.north east) {Jak napisa\'c prac\k{e} licencjack\k{a}};
    \end{tikzpicture}
  }
  \fancyfoot[C]{
    \begin{tikzpicture}[remember picture, overlay]
      \fill[acadnavy] ([yshift=14mm]current page.south west) rectangle ([yshift=0mm]current page.south east);
      \fill[acadgold] ([xshift=-20mm,yshift=5mm]current page.south) circle (4mm);
      \node[text=acadnavy, font=\bfseries\small] at ([xshift=-20mm,yshift=5mm]current page.south) {\thepage};
    \end{tikzpicture}
  }
  \renewcommand{\headrulewidth}{0pt}
  \renewcommand{\footrulewidth}{0pt}
}

% Styl: Wstęp i podstawy
\fancypagestyle{wstep}{
  \fancyhf{}
  \fancyhead[C]{
    \begin{tikzpicture}[remember picture, overlay]
      \fill[acadslate] ([yshift=-14mm]current page.north west) rectangle ([yshift=0mm]current page.north east);
      \node[text=white, anchor=west, font=\small] at ([xshift=15mm, yshift=-7mm]current page.north west) {\faIcon{book-open}};
      \node[text=white, anchor=east, font=\bfseries\small] at ([xshift=-15mm, yshift=-7mm]current page.north east) {Wprowadzenie};
    \end{tikzpicture}
  }
  \fancyfoot[C]{
    \begin{tikzpicture}[remember picture, overlay]
      \fill[acadslate] ([yshift=14mm]current page.south west) rectangle ([yshift=0mm]current page.south east);
      \fill[white] ([xshift=-20mm,yshift=5mm]current page.south) circle (4mm);
      \node[text=acadslate, font=\bfseries\small] at ([xshift=-20mm,yshift=5mm]current page.south) {\thepage};
    \end{tikzpicture}
  }
  \renewcommand{\headrulewidth}{0pt}
  \renewcommand{\footrulewidth}{0pt}
}

% Styl: Struktura pracy
\fancypagestyle{struktura}{
  \fancyhf{}
  \fancyhead[C]{
    \begin{tikzpicture}[remember picture, overlay]
      \fill[acadnavy] ([yshift=-14mm]current page.north west) rectangle ([yshift=0mm]current page.north east);
      \node[text=acadgold, anchor=west, font=\small] at ([xshift=15mm, yshift=-7mm]current page.north west) {\faIcon{sitemap}};
      \node[text=white, anchor=east, font=\bfseries\small] at ([xshift=-15mm, yshift=-7mm]current page.north east) {Struktura pracy};
    \end{tikzpicture}
  }
  \fancyfoot[C]{
    \begin{tikzpicture}[remember picture, overlay]
      \fill[acadnavy] ([yshift=14mm]current page.south west) rectangle ([yshift=0mm]current page.south east);
      \fill[acadgold] ([xshift=-20mm,yshift=5mm]current page.south) circle (4mm);
      \node[text=acadnavy, font=\bfseries\small] at ([xshift=-20mm,yshift=5mm]current page.south) {\thepage};
    \end{tikzpicture}
  }
  \renewcommand{\headrulewidth}{0pt}
  \renewcommand{\footrulewidth}{0pt}
}

% Styl: Źródła i bibliografia
\fancypagestyle{zrodla}{
  \fancyhf{}
  \fancyhead[C]{
    \begin{tikzpicture}[remember picture, overlay]
      \fill[acadforest] ([yshift=-14mm]current page.north west) rectangle ([yshift=0mm]current page.north east);
      \node[text=white, anchor=west, font=\small] at ([xshift=15mm, yshift=-7mm]current page.north west) {\faIcon{book}};
      \node[text=white, anchor=east, font=\bfseries\small] at ([xshift=-15mm, yshift=-7mm]current page.north east) {\'Zr\'od\l a i bibliografia};
    \end{tikzpicture}
  }
  \fancyfoot[C]{
    \begin{tikzpicture}[remember picture, overlay]
      \fill[acadforest] ([yshift=14mm]current page.south west) rectangle ([yshift=0mm]current page.south east);
      \fill[white] ([xshift=-20mm,yshift=5mm]current page.south) circle (4mm);
      \node[text=acadforest, font=\bfseries\small] at ([xshift=-20mm,yshift=5mm]current page.south) {\thepage};
    \end{tikzpicture}
  }
  \renewcommand{\headrulewidth}{0pt}
  \renewcommand{\footrulewidth}{0pt}
}

% Styl: Pisanie i redakcja
\fancypagestyle{pisanie}{
  \fancyhf{}
  \fancyhead[C]{
    \begin{tikzpicture}[remember picture, overlay]
      \fill[acadcopper] ([yshift=-14mm]current page.north west) rectangle ([yshift=0mm]current page.north east);
      \node[text=white, anchor=west, font=\small] at ([xshift=15mm, yshift=-7mm]current page.north west) {\faIcon{pen-fancy}};
      \node[text=white, anchor=east, font=\bfseries\small] at ([xshift=-15mm, yshift=-7mm]current page.north east) {Pisanie i redakcja};
    \end{tikzpicture}
  }
  \fancyfoot[C]{
    \begin{tikzpicture}[remember picture, overlay]
      \fill[acadcopper] ([yshift=14mm]current page.south west) rectangle ([yshift=0mm]current page.south east);
      \fill[white] ([xshift=-20mm,yshift=5mm]current page.south) circle (4mm);
      \node[text=acadcopper, font=\bfseries\small] at ([xshift=-20mm,yshift=5mm]current page.south) {\thepage};
    \end{tikzpicture}
  }
  \renewcommand{\headrulewidth}{0pt}
  \renewcommand{\footrulewidth}{0pt}
}

% Styl: Formatowanie
\fancypagestyle{formatowanie}{
  \fancyhf{}
  \fancyhead[C]{
    \begin{tikzpicture}[remember picture, overlay]
      \fill[acadplum] ([yshift=-14mm]current page.north west) rectangle ([yshift=0mm]current page.north east);
      \node[text=white, anchor=west, font=\small] at ([xshift=15mm, yshift=-7mm]current page.north west) {\faIcon{align-left}};
      \node[text=white, anchor=east, font=\bfseries\small] at ([xshift=-15mm, yshift=-7mm]current page.north east) {Formatowanie};
    \end{tikzpicture}
  }
  \fancyfoot[C]{
    \begin{tikzpicture}[remember picture, overlay]
      \fill[acadplum] ([yshift=14mm]current page.south west) rectangle ([yshift=0mm]current page.south east);
      \fill[white] ([xshift=-20mm,yshift=5mm]current page.south) circle (4mm);
      \node[text=acadplum, font=\bfseries\small] at ([xshift=-20mm,yshift=5mm]current page.south) {\thepage};
    \end{tikzpicture}
  }
  \renewcommand{\headrulewidth}{0pt}
  \renewcommand{\footrulewidth}{0pt}
}

% Styl: Obrona
\fancypagestyle{obrona}{
  \fancyhf{}
  \fancyhead[C]{
    \begin{tikzpicture}[remember picture, overlay]
      \fill[acadteal] ([yshift=-14mm]current page.north west) rectangle ([yshift=0mm]current page.north east);
      \node[text=white, anchor=west, font=\small] at ([xshift=15mm, yshift=-7mm]current page.north west) {\faIcon{user-graduate}};
      \node[text=white, anchor=east, font=\bfseries\small] at ([xshift=-15mm, yshift=-7mm]current page.north east) {Obrona pracy};
    \end{tikzpicture}
  }
  \fancyfoot[C]{
    \begin{tikzpicture}[remember picture, overlay]
      \fill[acadteal] ([yshift=14mm]current page.south west) rectangle ([yshift=0mm]current page.south east);
      \fill[white] ([xshift=-20mm,yshift=5mm]current page.south) circle (4mm);
      \node[text=acadteal, font=\bfseries\small] at ([xshift=-20mm,yshift=5mm]current page.south) {\thepage};
    \end{tikzpicture}
  }
  \renewcommand{\headrulewidth}{0pt}
  \renewcommand{\footrulewidth}{0pt}
}

% ============================================
% NAGLOWEK SEKCJI Z IKONA
% ============================================
\newcommand{\coloredsection}[3]{%
  \begin{tikzpicture}
    \node[
      fill=#1,
      text=white,
      font=\bfseries\large,
      minimum width=\textwidth,
      minimum height=11mm,
      inner sep=6pt
    ] (box) {#3};
    \node[text=white, anchor=west, font=\normalsize] at ([xshift=5mm]box.west) {#2};
  \end{tikzpicture}
  \vspace{4mm}
}

\newcommand{\sectionheader}[2]{%
  \begin{tikzpicture}
    \node[
      fill=#1,
      text=white,
      font=\bfseries\large,
      minimum width=\textwidth,
      minimum height=10mm,
      inner sep=5pt
    ] {#2};
  \end{tikzpicture}
  \vspace{3mm}
}

% ============================================
% BOX: DEFINICJA
% ============================================
\newtcolorbox{definicja}[1][]{%
  enhanced,
  breakable,
  before skip=4mm,
  after skip=4mm,
  colback=acadlightnavy,
  colframe=white,
  boxrule=0pt,
  borderline west={4pt}{0pt}{acadnavy},
  arc=0pt,
  left=6mm,
  right=5mm,
  top=4mm,
  bottom=4mm,
  fontupper=\small,
  title={\textcolor{acadnavy}{\faIcon{bookmark}~\bfseries Definicja\ifx&#1&\else: #1\fi}},
  coltitle=acadnavy,
  attach boxed title to top left={yshift=-2mm,xshift=4mm},
  boxed title style={colback=acadlightnavy,colframe=white,boxrule=0pt}
}

% ============================================
% BOX: PRZYKŁAD
% ============================================
\newtcolorbox{przyklad}[1][]{%
  enhanced,
  breakable,
  before skip=4mm,
  after skip=4mm,
  colback=acadlightforest,
  colframe=white,
  boxrule=0pt,
  borderline west={4pt}{0pt}{acadforest},
  arc=0pt,
  left=6mm,
  right=5mm,
  top=4mm,
  bottom=4mm,
  fontupper=\small,
  title={\textcolor{acadforest}{\faIcon{lightbulb}~\bfseries Przyk\l ad\ifx&#1&\else: #1\fi}},
  coltitle=acadforest,
  attach boxed title to top left={yshift=-2mm,xshift=4mm},
  boxed title style={colback=acadlightforest,colframe=white,boxrule=0pt}
}

% ============================================
% BOX: ZASADA
% ============================================
\newtcolorbox{zasada}[1][]{%
  enhanced,
  breakable,
  before skip=4mm,
  after skip=4mm,
  colback=acadlightgold,
  colframe=white,
  boxrule=0pt,
  borderline west={4pt}{0pt}{acadgold},
  arc=0pt,
  left=6mm,
  right=5mm,
  top=4mm,
  bottom=4mm,
  fontupper=\small,
  title={\textcolor{acadgold!80!black}{\faIcon{star}~\bfseries Zasada\ifx&#1&\else: #1\fi}},
  coltitle=acadgold!80!black,
  attach boxed title to top left={yshift=-2mm,xshift=4mm},
  boxed title style={colback=acadlightgold,colframe=white,boxrule=0pt}
}

% ============================================
% BOX: BŁĄD DO UNIKANIA
% ============================================
\newtcolorbox{blad}{%
  enhanced,
  breakable,
  before skip=4mm,
  after skip=4mm,
  colback=acadlightburgundy,
  colframe=white,
  boxrule=0pt,
  borderline west={4pt}{0pt}{acadburgundy},
  arc=0pt,
  left=6mm,
  right=5mm,
  top=4mm,
  bottom=4mm,
  fontupper=\small,
  title={\textcolor{acadburgundy}{\faIcon{exclamation-triangle}~\bfseries Cz\k{e}sty b\l\k{a}d}},
  coltitle=acadburgundy,
  attach boxed title to top left={yshift=-2mm,xshift=4mm},
  boxed title style={colback=acadlightburgundy,colframe=white,boxrule=0pt}
}

% ============================================
% BOX: WSKAZÓWKA
% ============================================
\newtcolorbox{wskazowka}{%
  enhanced,
  breakable,
  before skip=4mm,
  after skip=4mm,
  colback=acadlightteal,
  colframe=white,
  boxrule=0pt,
  borderline west={4pt}{0pt}{acadteal},
  arc=0pt,
  left=6mm,
  right=5mm,
  top=4mm,
  bottom=4mm,
  fontupper=\small,
  title={\textcolor{acadteal}{\faIcon{info-circle}~\bfseries Wskaz\'owka}},
  coltitle=acadteal,
  attach boxed title to top left={yshift=-2mm,xshift=4mm},
  boxed title style={colback=acadlightteal,colframe=white,boxrule=0pt}
}

% ============================================
% BOX: WAŻNE
% ============================================
\newtcolorbox{wazne}{%
  enhanced,
  breakable,
  before skip=4mm,
  after skip=4mm,
  colback=acadlightcopper,
  colframe=white,
  boxrule=0pt,
  borderline west={4pt}{0pt}{acadcopper},
  arc=0pt,
  left=6mm,
  right=5mm,
  top=4mm,
  bottom=4mm,
  fontupper=\small,
  title={\textcolor{acadcopper}{\faIcon{exclamation-circle}~\bfseries Wa\.zne}},
  coltitle=acadcopper,
  attach boxed title to top left={yshift=-2mm,xshift=4mm},
  boxed title style={colback=acadlightcopper,colframe=white,boxrule=0pt}
}

% ============================================
% BOX: KROK (NUMEROWANY)
% ============================================
\newcounter{stepcount}
\newcommand{\resetsteps}{\setcounter{stepcount}{0}}
\newtcolorbox{krok}[2][acadnavy]{%
  enhanced,
  before skip=4mm,
  after skip=4mm,
  colback=white,
  colframe=white,
  boxrule=0pt,
  borderline west={4pt}{0pt}{#1},
  arc=0pt,
  outer arc=0pt,
  left=6mm,
  right=5mm,
  top=4mm,
  bottom=4mm,
  fonttitle=\bfseries\color{black},
  title={\stepcounter{stepcount}\textcolor{#1}{\faIcon{chevron-right}}~Krok \thestepcount: #2},
  attach boxed title to top left={yshift=-2mm, xshift=0mm},
  boxed title style={colback=white,colframe=white,boxrule=0pt,left=0pt,right=0pt}
}

% ============================================
% BOX: CYTAT
% ============================================
\newtcolorbox{cytat}{%
  enhanced,
  breakable,
  before skip=4mm,
  after skip=4mm,
  colback=acadlightslate,
  colframe=white,
  boxrule=0pt,
  borderline west={3pt}{0pt}{acadslate},
  borderline east={3pt}{0pt}{acadslate},
  arc=0pt,
  left=8mm,
  right=8mm,
  top=4mm,
  bottom=4mm,
  fontupper=\small\itshape
}

% ============================================
% LISTY
% ============================================
\newlist{navylist}{enumerate}{1}
\setlist[navylist]{
  label={\textcolor{acadnavy}{\bfseries\arabic*.}},
  leftmargin=*,
  itemsep=3pt,
  topsep=3pt,
  parsep=0pt
}

\newlist{goldlist}{itemize}{1}
\setlist[goldlist]{
  label={\textcolor{acadgold}{$\blacktriangleright$}},
  leftmargin=*,
  itemsep=3pt,
  topsep=3pt,
  parsep=0pt
}

\newlist{forestlist}{itemize}{1}
\setlist[forestlist]{
  label={\textcolor{acadforest}{\faIcon{check}}},
  leftmargin=*,
  itemsep=3pt,
  topsep=3pt,
  parsep=0pt
}

\newlist{burgundylist}{itemize}{1}
\setlist[burgundylist]{
  label={\textcolor{acadburgundy}{\faIcon{times}}},
  leftmargin=*,
  itemsep=3pt,
  topsep=3pt,
  parsep=0pt
}

% ============================================
% TABELA - STYLE
% ============================================
\newcommand{\tableheadercolor}{acadnavy}
\newcolumntype{H}{>{\columncolor{\tableheadercolor}\color{white}\bfseries}c}
\newcolumntype{L}{>{\columncolor{\tableheadercolor}\color{white}\bfseries}l}

% ============================================
% PLACEHOLDER NA OBRAZEK
% ============================================
\newcommand{\screenshotplaceholder}[1]{%
  \begin{center}
  \begin{tikzpicture}
    \node[
      draw=acadslate!50,
      dashed,
      line width=1pt,
      minimum width=0.7\textwidth,
      minimum height=3.5cm,
      fill=acadlightslate
    ] {\textcolor{acadslate}{\textit{#1}}};
  \end{tikzpicture}
  \end{center}
}

\newcommand{\diagramplaceholder}[1]{%
  \begin{center}
  \begin{tikzpicture}
    \node[
      draw=acadnavy!50,
      dashed,
      line width=1pt,
      minimum width=0.65\textwidth,
      minimum height=4cm,
      fill=acadlightnavy
    ] {\textcolor{acadnavy}{\faIcon{project-diagram}~\textit{#1}}};
  \end{tikzpicture}
  \end{center}
}

% ============================================
% WSTAWIANIE OBRAZOW
% ============================================
\newcommand{\screenshot}[1]{%
  \begin{center}
  \includegraphics[width=0.75\textwidth]{#1}
  \end{center}
}

\newcommand{\screenshotcaption}[2]{%
  \begin{center}
  \includegraphics[width=0.75\textwidth]{#1}\\[2mm]
  \textcolor{acadslate}{\small\textit{#2}}
  \end{center}
}

% ============================================
% SEPARATOR SEKCJI
% ============================================
\newcommand{\sectionbreak}{%
  \begin{center}
  \vspace{2mm}
  \textcolor{acadgold}{$\blacklozenge$\hspace{3mm}$\blacklozenge$\hspace{3mm}$\blacklozenge$}
  \vspace{2mm}
  \end{center}
}

% ============================================
% IKONY STATUSU
% ============================================
\newcommand{\statusok}{\textcolor{acadforest}{\faIcon{check-circle}}}
\newcommand{\statuswarn}{\textcolor{acadgold}{\faIcon{exclamation-circle}}}
\newcommand{\statuserror}{\textcolor{acadburgundy}{\faIcon{times-circle}}}

% ============================================
% TYPOGRAFIA
% ============================================
\hyphenpenalty=5000
\exhyphenpenalty=5000
\tolerance=2000
\raggedbottom
\hypersetup{pdfnewwindow=true}

\pagestyle{main}
